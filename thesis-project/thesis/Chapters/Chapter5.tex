\chapter{Threats to Validity}
\label{Chapter5}

It is crucial to acknowledge potential threats to the validity of an experiment. Firstly, it's important to clarify that the evaluation of readability results, both before and after modification, relies on a heuristic-based tool. While this evaluation may not be mathematically precise, it holds significant statistical value. This aspect is pivotal as much of the experiment hinges on these heuristics; hence, any inaccuracies in them could significantly impact the results. Another potential threat to validity arises from the experiment's reliance on a relatively small sample of projects. Consequently, we lack sufficient data to deem the results statistically representative. However, this serves as a solid starting point, as it underscores the genuine potential of such models, albeit in a limited capacity. \newline \newline A potential threat to the validity of the study lies in the fact that a significant portion of the entire analysis relies on manual validation. This manual evaluation played a crucial and substantive role, given that the model, at the time of publication, only has the capability of outputting in lowercase. Therefore, this phase has played an important part in the study, and it is possible that this validation may have introduced errors that could have influenced the results. \newline Another threat to the validity of the study arises from the fact that the results regarding the correctness of the methods were obtained through the analysis of the test suite. This may not be sufficient to guarantee the correctness of the code, as such a test suite may not cover all possible scenarios in which a bug can be introduced.
